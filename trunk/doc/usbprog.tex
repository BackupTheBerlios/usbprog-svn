%% Erl�uterungen zu den Befehlen erfolgen unter
%% diesem Beispiel.                
\documentclass{scrartcl}
\usepackage[latin1]{inputenc}
\usepackage[T1]{fontenc}
\usepackage[ngerman]{babel}
\usepackage{amsmath}

\title{usbprog - Mikrocontroller Entwicklungskit}
\author{Benedikt Sauter}
\date{07. Februar 2007}
\begin{document}

\maketitle
\tableofcontents
\section{Was ist usbprog?}

Haupts�chlich ist usbprog eine universelle Hardware die auf 
der Computer Seite eine USB Schnittstelle hat. Auf
dem Adapter eine SPI Schnittstelle
oder alternativ 4 Portleitungen und eine RS232 Schnittstelle.

�ber einen einfachen Update Mechanismus, kann man direkt
�ber die bestehende USB Verbindung die Software 
auf dem Adapter austauschen. Die geht bequem �ber ein
grafisches Programm welches es f�r Linux und Windows gibt.

Da alle Pl�ne und Programme unter einer freien Lizenz stehen,
kann man eigene Firmware Versionen f�r usbprog entwicklen.


\subsection{Firmware Versionen}

\subsubsection{AVRISP mkII Klon}
	Dieser Klon verh�lt sich 1:1 wie ein AVRISP mkII. Bisher wurde
	er erfolgreich mit AVR Studio 4 und avrdude getestet.

\subsubsection{usbprog - AVR Porgrammierer und RS232 Schnittstelle �ber USB}
	Mit dieser kombination arbeiten wohl die meisten Entwickler.
	Normalerweise sieht es am Tisch des Entwicklers dann so aus,
	das ein Parallelport-Kabel und ein Serielles Kabel vom PC wegeht. Dazu
	wird dann meistens noch f�r das Zielboard eine Versorgungsspannung
	ben�titg. Mit dieser Firmwareversion sparrt man sich zwei von den
	drei Kabeln.

	Die serielle Schnittstelle wird aktuell nur von Linux (als /dev/ttyUSBx) unterst�tzt.

\subsubsection{usbprog - JTAG Interface, AVR On-Chip-Debugger}
	Diese Version ist gerade noch in der Entwicklung. Man kann dann ohne extra Hardwareaufwand
	einfach Atmels AVR Controller debuggen. 


\subsection{Hardware}

\LaTeX{} ist auch ohne Formeln sehr n�tzlich und
einfach zu verwenden. Grafiken, Tabellen,
Querverweise aller Art, Literatur- und
Stichwortverzeichnis sind kein Problem.

Formeln sind etwas schwieriger, dennoch hier ein
einfaches Beispiel.  Zwei von Einsteins
ber�hmtesten Formeln lauten:
\begin{align}
E &= mc^2                                  \\
m &= \frac{m_0}{\sqrt{1-\frac{v^2}{c^2}}}
\end{align}
Aber wer keine Formeln schreibt, braucht sich
damit auch nicht zu besch�ftigen.


\subsection{Firmware installieren}


\subsection{Unterst�zte Software}


\end{document}
